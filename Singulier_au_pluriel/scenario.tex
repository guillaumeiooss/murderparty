% Créé par Giooss, en utilisant le template Latex de Martin Bodin
% Document sous licence CC BY-NC-SA

% Créé par Martin Bodin (2011).
% Document sous licence CC BY-NC-SA

\documentclass{article}
%\documentclass{scrartcl}

\usepackage{ifxetex}
\ifxetex
\usepackage{xunicode,fontspec,xltxtra}
\else
\usepackage[utf8x]{inputenc}
\usepackage[T1]{fontenc}
\usepackage{amsmath, amsthm}
\usepackage{amsfonts, amssymb}
\fi

\usepackage[francais]{babel}
\usepackage{lmodern}
\usepackage{stmaryrd}
\usepackage{graphicx}
\usepackage[nottoc, notlof, notlot]{tocbibind}
\usepackage[dvipsnames]{pstricks}
\usepackage{pst-circ, pst-plot, pstricks-add}
\usepackage{array}
\usepackage{url}
%\usepackage{verse}
\usepackage[colorlinks,linkcolor=black]{hyperref}
\usepackage{ifthen}
\usepackage{longtable, rotating}
%\usepackage{fancyhdr}
\usepackage{fancybox, framed}
\usepackage{textcomp}
\usepackage{marvosym}
%\usepackage{bbding}
%\usepackage{a4wide}
\usepackage{geometry}
%\usepackage{soul}
\usepackage{lettrine}
%\usepackage{yfonts}
\usepackage{oldgerm}
\usepackage{enumerate}
\usepackage{tikz}
\usepackage{dictsym}
\usepackage{pifont}

\ifxetex
\newfontfamily\timesfont[Ligatures=TeX]{Times New Roman}
\setmainfont[Mapping=tex-text, Ligatures={Contextual, Common, Historical, Rare, Discretionary}, Numbers={OldStyle}]{Linux Libertine O}
\fi

%\newcommand{\enluminure}[2]{\lettrine[lines=3]{\small \initfamily #1}{#2}}

\usetikzlibrary{trees}
\usetikzlibrary{arrows,shapes,automata,petri}
\usetikzlibrary{fit}
\usetikzlibrary{calc,decorations.pathmorphing,patterns}


\geometry{
	includeheadfoot,
	margin = 2.5cm,
	top = 1.5cm,
	bottom = 1.5cm
}

\newcommand{\ds}{\displaystyle}

\renewcommand{\ge}{\geqslant}
\renewcommand{\le}{\leqslant}
\renewcommand{\preceq}{\preccurlyeq}
\renewcommand{\succeq}{\succcurlyeq}

\newcommand{\Numero}{\No}
\newcommand{\numero}{\no}

\newcommand{\fixme}{\textbf{FIXME}}

\makeatletter

\newcommand{\defineNewPlayer}[2]{
	\@namedef{couleur#1}{#2}
}

\newcommand{\getPlayerColor}[1]{%
	\@nameuse{couleur#1}%
}

\makeatother

% Des commandes pratiques pour générer le document.
\newcommand{\player}[2]{%
	\ifthenelse{\equal{\forplayer}{y}}{%
		\ifthenelse{\equal{\theplayer}{#1}}%
		{#2}{}%
	}{\begin{barv}[\getPlayerColor{#1}]{2pt}{10pt}#2\end{barv}}%
}
\newcommand{\mj}[1]{%
	\ifthenelse{\equal{\forplayer}{n}}{#1}{}%
}
% Ici suit une commande plus complexe, car plus générale.
\makeatletter

\newcommand{\@beginColor}[3][black]{%
	\ifthenelse{\equal{\forplayer}{n}}{%
		\begin{barv}[#1]{#2}{#3}%
	}{}%
}

\newcommand{\@endColor}{%
	\ifthenelse{\equal{\forplayer}{n}}{%
		\end{barv}%
	}{}%
}


\newcommand{\ignore}[1]{}
\newcommand{\@ident}[3]{%
%	\ifthenelse{\equal{\manyColored}{y}}{#1}{%
%		\marginpar{%
%			#1%
%			\vspace{2cm}%
%			#2%
%		}%
%	}%
	#1%
	\ifthenelse{\equal{\forplayer}{n}}{\@beginColor{0pt}{10pt}}{}%
	#3%
	\ifthenelse{\equal{\forplayer}{n}}{\@endColor}{}%
	#2%
%	\ifthenelse{\equal{\manyColored}{y}}{#2}{%
%		\marginpar{%
%			#1%
%			\vspace{5pt}%
%			#2%
%		}%
%	}%
}

\def\@ouverture#1#2{%
\ifthenelse{\equal{\forplayer}{y}}{}{%
\ifthenelse{\equal{\manyColored}{y}}{\@beginColor[#1]{1pt}{0pt}}{%
\hspace{-1cm}\hspace{-#2mm}\parbox[c][1pt][t]{0pt}{
\begin{tikzpicture}
	\node (a) {};
	\node (b) [right of = a, node distance = 16cm] {};
	\node (c) [below of = a, node distance = 2cm] {};
	\draw [very thick, color = #1] (a.center) -- (b);
	\draw [very thick, color = #1] (a.center) -- (c);
\end{tikzpicture}
}\vspace{-3.2mm}\par%
}%
}%
}
\def\@fermeture#1#2{%
\ifthenelse{\equal{\forplayer}{y}}{}{%
\ifthenelse{\equal{\manyColored}{y}}{\@endColor}{%
\hspace{-1cm}\hspace{-#2mm}\parbox[c][1pt][b]{0pt}{
\begin{tikzpicture}
	\node (a) {};
	\node (b) [right of = a, node distance = 16cm] {};
	\node (c) [above of = a, node distance = 1cm] {};
	\draw [very thick, color = #1] (a.center) -- (b);
	\draw [very thick, color = #1, dashed] (a.center) -- (c);
\end{tikzpicture}
}\vspace{-3.2mm}\par%
}%
}%
}

\def\players@parse#1#2[#3][#4]{%
% #1 :  Suite de \@ouverture
% #2 :  Suite de \@fermeture
% #3 :  Commande à appeler dans le cas d’une réponse négative (≃ réponse précédente).
% #4 :  Argument (sous forme de numéro de joueur) lu actuellement.
	\ifthenelse{\equal{\theplayer}{#4}}{%
		\players@yes{\@ouverture{\getPlayerColor{#4}}{#4}#1}{#2\@fermeture{\getPlayerColor{#4}}{#4}}%
	}{%
		#3{\@ouverture{\getPlayerColor{#4}}{#4}#1}{#2\@fermeture{\getPlayerColor{#4}}{#4}}%
	}%
}

\def\players@no#1#2{%
	\@ifnextchar[{\players@parse{#1}{#2}[\players@no]}{\ignore}%
}

\def\players@yes#1#2{%
	\@ifnextchar[{\players@parse{#1}{#2}[\players@yes]}{\@ident{#1}{#2}}%
}

\def\players{%
	\ifthenelse{\equal{\forplayer}{y}}{%
		\players@no{}{}%
	}{%
		\players@yes{}{}%
	}%
}

% \players{…} est quasi-équivalent à \mj{…}.
% \players[i]{…} est équivalent à \player{i}{…}
% \players[i][j][k]{…} va créer du contenu uniquement pour les joueurs i, j et k (et les MJ bien sûr).

\makeatother
%\fixme :  Ces commandes posent des problèmes pour toutes les sections, footnote, etc. :S

\newcommand{\colorForMJ}[2]{%
	\ifthenelse{\equal{\forplayer}{y}}{#2}{%
		\textcolor{\getPlayerColor{#1}}{#2}%
	}%
}
\newcommand{\synopsisPerso}[3]{%
\paragraph{}{
\textbf{\fcolorbox{\getPlayerColor{#1}}{white}{#2}}\hspace{10pt}%
{#3}}%
}

\newenvironment{changemargin}[2]{\begin{list}{}{%
\setlength{\topsep}{0pt}%
\setlength{\leftmargin}{0pt}%
\setlength{\rightmargin}{0pt}%
\setlength{\listparindent}{\parindent}%
\setlength{\itemindent}{\parindent}%
\setlength{\parsep}{0pt plus 1pt}%
\addtolength{\leftmargin}{#1}%
\addtolength{\rightmargin}{#2}%
}\item }{\end{list}}
\reversemarginpar
%\pagestyle{fancy}
%\fancyhf{}
%\renewcommand{\headrulewidth}{0pt}
%\lhead{}
%\lfoot{}

\makeatletter
\newenvironment{barv}[3][black]{%
% #2 largeur du trait
% #3 distance entre le trait et le texte
	\def\FrameCommand{{\color{#1}\vrule width #2}
	\hspace{#3}}%
	\MakeFramed {\advance \hsize -\width \FrameRestore }%
}{%
    \endMakeFramed%
}
\makeatother


\definecolor{LightButter}{rgb}{0.98,0.91,0.31}
\definecolor{LightOrange}{rgb}{0.98,0.68,0.24}
\definecolor{LightChocolate}{rgb}{0.91,0.72,0.43}
\definecolor{LightChameleon}{rgb}{0.54,0.88,0.20}
\definecolor{LightSkyBlue}{rgb}{0.45,0.62,0.81}
\definecolor{LightPlum}{rgb}{0.68,0.50,0.66}
\definecolor{LightScarletRed}{rgb}{0.93,0.16,0.16}
\definecolor{Butter}{rgb}{0.93,0.86,0.25}
\definecolor{Orange}{rgb}{0.96,0.47,0.00}
\definecolor{Chocolate}{rgb}{0.75,0.49,0.07}
\definecolor{Chameleon}{rgb}{0.45,0.82,0.09}
\definecolor{SkyBlue}{rgb}{0.20,0.39,0.64}
\definecolor{Plum}{rgb}{0.46,0.31,0.48}
\definecolor{ScarletRed}{rgb}{0.80,0.00,0.00}
\definecolor{DarkButter}{rgb}{0.77,0.62,0.00}
\definecolor{DarkOrange}{rgb}{0.80,0.36,0.00}
\definecolor{DarkChocolate}{rgb}{0.56,0.35,0.01}
\definecolor{DarkChameleon}{rgb}{0.30,0.60,0.02}
\definecolor{DarkSkyBlue}{rgb}{0.12,0.29,0.53}
\definecolor{DarkPlum}{rgb}{0.36,0.21,0.40}
\definecolor{DarkScarletRed}{rgb}{0.64,0.00,0.00}
\definecolor{Aluminium1}{rgb}{0.93,0.93,0.92}
\definecolor{Aluminium2}{rgb}{0.82,0.84,0.81}
\definecolor{Aluminium3}{rgb}{0.73,0.74,0.71}
\definecolor{Aluminium4}{rgb}{0.53,0.54,0.52}
\definecolor{Aluminium5}{rgb}{0.33,0.34,0.32}
\definecolor{Aluminium6}{rgb}{0.18,0.20,0.21}

\pgfdeclarelayer{foreground} 
\pgfdeclarelayer{background} 
\pgfsetlayers{background,main,foreground} 



% --- Réglages pour générer les feuilles de persos.
\newcommand{\forplayer}{n} % n => pour le mj / y => tient compte de la commande suivante
\newcommand{\theplayer}{1} % Numéro du joueur dont la fiche est généré
\newcommand{\manyColored}{n}


% --- Liste des personnages
\defineNewPlayer{1}{Red}			% Dominique Magnus (Savant fou)
\defineNewPlayer{2}{Blue}			% Nathan Leore (Trafiquant)
\defineNewPlayer{3}{OliveGreen}		% Lance Rhenato (Etudiant)
\defineNewPlayer{4}{Cyan}			% Jean Storm (Troll)
\defineNewPlayer{5}{Brown}			% Lake Helm - futur (Visiteur du futur)
\defineNewPlayer{6}{Yellow}			% Elza Caro (magicienne)
\defineNewPlayer{7}{Plum}			% Lake Helm (Inspecteur)
\defineNewPlayer{8}{Gray}			% Grégoire Bastre (Muscle drogué)
\defineNewPlayer{9}{DarkSkyBlue}	% Delphe Umé (prophète)
\defineNewPlayer{10}{BlueViolet}	% Redrick Illonois (Highlander)
\defineNewPlayer{11}{Rhodamine}		% Robert Dudestin (Catalyse)
\defineNewPlayer{12}{Violet}		% Nathan Lithe (Chimiste)


% --- Noms des personnages
\newcommand{\nmPlayerI}{Dominique Magnus}
\newcommand{\nmPlayerII}{Nathan Leore}
\newcommand{\nmPlayerIII}{Lance Rhenato}
\newcommand{\nmPlayerIV}{Jean Storm}
\newcommand{\nmPlayerV}{Luc Estravos}
\newcommand{\nmPlayerVI}{Elza Caro}
\newcommand{\nmPlayerVII}{Lake Helm}
\newcommand{\nmPlayerVIII}{Grégoire Bastre}
\newcommand{\nmPlayerIX}{Delphe Umé}
\newcommand{\nmPlayerX}{Redrick Illonois}
\newcommand{\nmPlayerXI}{Robert Dudestin}
\newcommand{\nmPlayerXII}{Nathan Lithe}


\newcommand{\pageForPlayer}[4]{%
\player{#1}{
	%\mj{\newpage}%
	\section{Il était une fois... ton personnage~: #2}
	\begin{description}
		#3
	\end{description}
\par
	\paragraph{Description du personnage.}
	{#4}
}}

% -----------------------------------------------------------------------------

\title{Du singulier au pluriel}
\author{Guillaume Iooss}
\date \today

\begin{document}


\mj{\maketitle}

% *** Remarques sur les commandes latex de Martin:
% - "\mj{ ... }"														=> permet qu'un texte ne soit vu que par le MJ.
% - "\players[n1][n2] ... [nk] { ... }"									=> permet qu'un texte ne soit vu que par les joueurs n1, ... , nk
% - Tout ça s'imbrique comme on veut (et tout est visible par le MJ dans tous les cas)
% - "\pageForPlayer{n}{Nom_joueur}{description_en_item}{ ... }			=> Génère une page de description du joueur numéro "n", ayant pour nom "Nom_joueur", etc etc


% ********************************************************
% *** Description de l'univers - version complète (MJ) ***
% ********************************************************
\mj{ \section{Résumé complet de la murder pour MJ}

	\subsection{Contexte historique de la murder}
	
	\par Cette murder se passe dans un univers de SF classique, où l'humanité est bien avancé dans sa conquête spatiale. A l’occurrence, elle fait partie d'une fédération d'empires, étalée sur une portion conséquente de la galaxie, avec des vaisseaux voyageant à des vitesses supra-luminiques qui communiquent entre les différentes planètes. Niveau aliens, ces derniers existent, mais ne se mixent pas fréquement avec les autres races (ce qui fait que leur occurrence est plutôt rare).
	
	\par Le lieu de la murder se passe dans une station de relai commerciale (nommée \emph{Aldébaran-4}), situé dans l'espace profond, et qui est un point de transit entre plusieurs quadrants de la galaxie. Une IA automatisée contrôle le bon fonctionnement de la station, et des officiers (majoritairement de douane) sont stationnés dans cette dernière.
	
	
	\paragraph{Celui qui contrôle l'épice\dots} Dans cet univers existe une drogue, nommée \emph{l'épice}. Comme son nom l'indique\footnote{Si vous ne comprenez pas la référence, allez lire le premier tome de Dune}, la consommation de cette drogue entraîne des visions du futur. Prise en continu et en petite quantité, il devient possible de deviner les mouvements d'un adversaire une seconde à l'avance, ce qui est un boost assez significatif pour tout soldat. Prise en quantité normale, elle étourdit pendant quelques minutes la personne en consommant et déclenche une vision du futur (du type "flash d'informations").
	
	\par Du fait que les effets de cette drogue sont trop puissants, cette drogue a été interdite dans tout l'empire et les ingrédients de sa fabrication ont été hautement réglementés. L'utilisation de cette drogue est néanmoins autorisé au confin de l'administration (typiquement pour des grands généraux ou soldat en temps de guerre, ou des politiciens et ambassadeurs en temps de paix).
	
	\par Afin de lutter contre la fabrication et propagation de l'épice, l'empire a fait courir le bruit que cette drogue était dangereuse, du fait des \emph{paradoxes temporels} qu'elle peut provoquer. Si une vision ne s'accomplit pas (du fait de l'interférence de celui qui a eu la vision), le paradoxe temporaire pourrait effacer la personne en question de l'histoire, si la causalité n'est pas rétablie (cela étant présenté comme une mesure de défense de l'univers, qui évite ce dernier d'exploser au moindre pépin). En vérité, c'est de la pure propagande, et l'univers (ou plutôt le multi-vers) se fiche de la présence de paradoxes ou non. Il n'y a donc aucun risque à provoquer un paradoxe.
	
	\par Un des effets secondaires de l'épice est qu'un flux de particules temporelles (des \emph{tachyons}) est généré à chaque consommation de la drogue. Du coup, afin de traquer les consommation, des capteurs ont été installés par défaut dans toutes les planètes et stations de l'empire.
	
	\par Une police temporelle a été également mise en place, chargée de traquer les consommateurs d'épice et de neutraliser tout potentiel de paradoxe. La procédure normale consiste à mettre en quarantaine la zone où la consommation s'est produite, de trouver le consommateur et d'effacer sa mémoire avant qu'une fuite d'information future se produise. Dans le pire des cas, la force létale est autorisée (sous réserve qu'elle soit justifiée comme étant la seule option restante, les règles sont similaires à celles des forces de police).
	
	\par La zone est descellée uniquement sous ordre de l'officier de la police temporelle dépêché, et lorsque la \emph{situation sera résolue} d'après l'IA. L'IA a pour instruction d'effacer systématiquement toute mémoire ou information pouvant sortir de la zone scellée, pour éviter tout piratage d'information future. Elle ne peut donc qu'intervenir de manière limitée dans l'enquête.
	
	
	\paragraph{La murder: alerte temporelle à Aldébaran-41} Au début de la partie, l'IA administrant la station relai Aldébaran-41 a détecté un flux de tachyon considérable venant d'une de ses sections. Elle applique dont le protocole par défaut et met en quarantaine cette partie. Fort heureusement, un inspecteur temporel (nommé \emph{\nmPlayerVII}) est déjà sur place ainsi qu'un collègue bio-chimiste (nommé \emph{\nmPlayerXII}) et un garde du corps (nommé \emph{\nmPlayerVIII}).
	
	\par Les joueurs correspondent aux personnes se trouvant dans la partie mise sous quarantaine, et ils sont les seuls à se trouver dans cette portion du vaisseau. L'IA de la station peut communiquer de manière limitée, et des serviteurs robotisés s'occupent de leur besoins immédiat.
	
	% ==========================================================
	\subsection{La singularité temporelle et des multiples versions de personnages}
	
	\par En réalité, l'histoire est (bien entendu) beaucoup plus compliquée\dots
	\begin{itemize}
		\item En fait, le flux de tachyon n'a pas été produite du fait d'une consommation d'épice de plusieurs personnages\dots mais par une anomalie temporelle qui a fait que plusieurs personnages venant du futur se sont retrouvés à ce moment précis. Ainsi, certains personnages se retrouvent avec plusieurs versions d'eux-même. Notamment \nmPlayerI, \nmPlayerII, \nmPlayerIII, \nmPlayerIV ~et \nmPlayerVI ~sont originellement la même personne, dans cet ordre chronologique, mis à part \nmPlayerVI ~ qui vient d'une réalité alternative. Des pertes de mémoires (dû aux événements de la murder dans une itération temporelle passée) ont été impliqués, ce qui fait que certains personnages futurs ne reconnaissent pas leur incarnation passée.
		
		\item A côté de ça, \nmPlayerV (visiteur du futur) est une version future de \nmPlayerVII (l'inspecteur). Il est revenu dans le temps pour empécher une fin apocalyptique du monde. Le problème est que l'inspecteur croit que les paradoxes doivent être évités à tout prix, du coup, ça va être compliqué de le convaincre.
		
		\item La-dite fin du monde est provoquée par \nmPlayerX (l'highlander de l'espace). Ce dernier est un mutant humain qui a pour capacité d'absorber les pouvoirs d'individus ayant une mutation similaire, et de les combiner, dans le but de devenir un dieu vivant. Ce dernier risque de tomber sur \nmPlayerIX (la prophète psi) qui a des pouvoirs naturels similaire à l'épice, et qui peut potentiellement lui donner des pouvoirs de prédiction le rendant invincible. Notez que l'absorbtion des pouvoirs de \nmPlayerVI ne marche pas, vu que ces derniers ont été appris (par opposition d'être innés).
		
		\item \nmPlayerIX (la psychique) est une humaine ayant des capacités de prédiction et qui a été capturée et étudiée par l'empire (ce qui a donné lieu à la création de l'épice, sur la base de l'étude de sa composition sanguine). Elle est actuellement en cavale et essaye de rester incognito. Elle a comme autre capacité de provoquer des visions, dont elle peut décider le contenu, chez quelqu'un. Cette dernière capacité est limitée, mais peut se rechargée par consommation d'épice. C'est cette capacité non connue par l'empire et qui l'a aidé à s'échapper. Elle a donc pour objectif principal de cacher ses capacités, ce qui est difficile du fait des visions qu'elle a régulièrement, ce qui provoque un flux de tachyon.
		
		\item Par ailleurs, \nmPlayerII (le trafiquant) a VRAIMENT de l'épice sur lui et était sensé l'échanger avec \nmPlayerVIII (l'agent de sécurité qui deale durant son temps libre). L'épice était planqué dans un paquet de farine, qui a été "emprunté" par \nmPlayerIII (l'étudiant fauché) pour faire un gâteau (dont le début de la cuisson sera lancé en début de partie). Du coup, les personnes mangeant le gâteau plus tard dans la partie consommeront de l'épice et auront des visions associées (au grand malheur de l'inspecteur).
		
		\item \nmPlayerXII (le chimiste) est un collègue de l'inspecteur et agent de l'empire dans le secret des rumeurs concernant l'épice et les paradoxes (du fait qu'il a de la famille proche dans la haute administration de l'empire). Du coup, il essayera de maintenir la rumeur que les paradoxes, c'est mal pour maintenir la propagande de l'empire. Il a plusieurs tests chimiques à poursuivre, mais demandant des sujets. Par exemple, il a un vaccin en cours de conception pouvant potentiellement immuniser une personne face aux effets de l'épice, mais il a besoin de cobayes ayant déjà consommé de l'épice avant de pouvoir le développer. Il possède également une drogue permettant d'effacer la mémoire des dernières 48h d'une personne.
		
		\item \nmPlayerXI (le catalyseur) est juste une personne normale\dots mais qui apparemment qui s'est pris la fameuse malédiction chinoise "Que votre vie soit intéressante". Du coup, tout évenement aléatoire pouvant arriver à quelqu'un lui arrivera probablement, et tous les personnages (notamment les plus dangereux) vont l'identifier comme étant une personne d'intérêt.
		
		\item \nmPlayerVI (la magicienne) est simplement une étudiante en magie dimensionnelle de l'univers parallèle d'à côté, qui a eu la malencontreuse idée d'éternuer en plein milieu d'un lancement d'un sort compliqué, et s'est retrouvée dans la station au moment de l'anomalie. Elle a pour objectif de se planquer, en attendant avoir récupérer suffisamment de mana pour concevoir et lancer un sort pour retourner dans sa dimension. Elle a potentiellement moyen de détecter les utilisations de pouvoir psy de \nmPlayerIX et peut y résister. Elle a également des visions prophétiques occasionnelles (vu qu'elle a de la magie). Notez que ses pouvoirs ont été acquis au cours de sa vie (et non pas innés, comme la psychique). Du coup, si \nmPlayerX tente d'absorber son génome, il découvrira qu'elle est une humaine normale.
		
		\item \nmPlayerI (le savant fou) a pour projet de devenir immortel. La solution qu'il a trouvé (et retenu) est de transférer sa personnalité dans son corps du passé afin de gagner quelques décennies de plus à vivre... puis d'itérer à l'infini, en accumulant à chaque fois plus de connaissance et d'expérience. Vu qu'il n'y a pas de paradoxe possible, ça marche en pratique. Du coup, le savant fou est la vrai cause de l'anomalie temporelle qui a été causé dans la station, qui a (à son tour) provoqué l'arrivée d'autres voyageurs random à ce lieu/temps précis (le troll, la magicienne et le visiteur du futur).
		
		\item \nmPlayerIV (le troll) est le résultat du savant fou ayant réussi son plan et ayant vécu de nombreux millénaires. Il a donc vécu cette situation de nombreuses fois, mais avec des variantes. Il connaît donc de multiples versions des archétypes de chaque persos, et connait plusieurs cheminements de l'histoire (pour les avoir vécu). Il a des fois des surprises. Notamment c'est la première fois qu'il rencontre la magicienne.
		
		A force de boucler et de vivre éternellement, il est devenu fou et s'amuse donc à provoquer des situations uniques (et humouristiques de son point de vue) afin de se distraire de sa longue vie. De plus, il a trouvé un moyen alternatif de vivre indéfiniment (en s'isolant du flux temporel de n'importe quel monde parallèle dans lequel il se trouve), du coup peut agir comme il le souhaite en toute impunité, sans risque de se faire tuer.
		
		\item \nmPlayerII (le trafiquant) est le résultat de \nmPlayerIII (l'étudiant), après une tentative ratée du savant fou de transférer sa personnalité lors d'une précédente itération de la boucle temporelle. Notamment, leur noms sont des anagrammes l'un de l'autre. Notez que \nmPlayerI a des souvenirs de son itération en tant que \nmPlayerII et \nmPlayerIII, juste que du fait de son effacement de mémoire en tant que \nmPlayerII, il pense n'avoir été qu'une fois dans cette station et mélange ses deux expériences.
		
		En gros, ça a eu les conséquences suivantes: le trafiquant s'est retrouvé projeté dans le passé (quelques années) et a dû devenir trafiquant pour notamment contrecarrer ses problèmes de manque de papier et d'argent. Le trafiquant a également eu une perte de mémoire (du à un des processus fait par le savant fou), mais sait qu'il a voyagé dans le passé et pense (sans être sûr) que \nmPlayerIII est son incarnation précédente, et qu'il s'est passé un truc. Notez que le trafiquant croit à ces histoires de paradoxes, et donc va essayer de se démerder pour que \nmPlayerIII se fasse également projeter dans le passé (d'un moyen ou d'un autre)
	\end{itemize}
	
	% IA de la station
	\paragraph{IA de la station} A côté de tout ça, l'IA de la station a un accès limité à la zone (pour éviter la propagation de toute information potentiellement paradoxale). Notamment, elle ne peut pas surveiller les personnes à l'intérieur de la zone de quarantaine (et même efface systématiquement toute information à ce propos), et a également effacé sa mémoire des dernières minutes avant que le paradoxe se produise (vu que cette mémoire peut contenir des informations paradoxales)
	
	
	% ==========================================================
	\subsection{Systèmes de jeu}
	
	\paragraph{Donneur d'ordre de la fin de l'isolation} L'IA gérant la station a pour instruction de desceller la zone une fois la situation résolue, càd une fois que les paradoxes ont été désamorcés. Vu que l'IA ne peut pas savoir quand la situation est résolue, il y a une hiérarchie de personne pouvant donner l'ordre de desceller la station, et donc de mettre fin à la partie. Dans l'ordre (vu qu'il peut avoir des meurtres), ces personnes sont: \nmPlayerVII (l'inspecteur), puis \nmPlayerXII (le chimiste), puis \nmPlayerVIII (l'agent de sécurité). Si aucun des 3 n'est disponible, un vote (démocratique) parmi les survivants sera fait pour décider de leur représentant.
	
%	\paragraph{Gestion du combat et armement} Il y a trois niveaux d'attaque:
%	\begin{itemize}
%		\item Faible: Blesse légèrement un joueur (ou assomme), l'incapacitant temporairement, mais lui laissant aucune séquelle postérieure.
%		\item Fort: Blesse gravement, incapacitant un joueur tout en laissant des (potentielles) séquelles. Peut aller jusqu'à le tuer si aucun soin n'est prodigué.
%		\item Létale: Tue instantanément un joueur.
%	\end{itemize}
%	\par Il est possible de réduire la puissance d'une attaque. Typiquement, si un joueur a la possibilité de faire une attaque léthale (par exemple), il peut décider de réduire la puissance de son attaque à faible ou fort.
	
%	\par Par défaut, une attaque à main nue est de force "Faible". Un entraînement au combat peut monter cette attaque à main nue d'un cran. Une arme (typiquement blanche) est généralement une attaque forte. Un phaseur peut aller jusqu'à une attaque létale. La consommation d'épice peut monter d'un cran la dangerosité d'une attaque.
	
	% Attaque surprise
%	\par Quand un joueur est attaqué et le sait, il a la possibilité de faire une réaction (typiquement se défendre, pousser un grand cri, \dots). L'exception est si l'attaque est une attaque surprise. Càd, l'attaque a été faite hors du champs de vision de la cible ET que l'attaquant n'a pas la main sur une de ses armes. Si la cible a un entraînement martial, elle ne peut pas se faire prendre par surprise (vu que ses poings comptent comme une arme au corps à corps).
	
	
	\paragraph{Visions du futur} Les visions du futur sont déclenchés une fois un certain nombre de conditions sont remplies (par exemple, une personne consomme de l'épice, ou un pouvoir d'un personnage se déclenche). Niveau mécanisme, ces visions sont préécrites sur des bouts de papier (préparés à l'avance) qui sont distribués à un joueur lors de la vision. Un joueur ayant une vision du futur ne peut pas réagir aux simulations extérieures pendant 1 minute entière (ce qui est un moyen de détecter une vision, et ce qui peut être une manière idéale de prendre quelqu'un par surprise), mais a quelques secondes d'avertissement au préalable (le MJ qui lui indique subtilement qu'une vision arrive). Un ordre des visions est imposé (= progression dans la partie), juste, le receveur de la vision n'est pas déterminé à l'avance.
	
	\par Au niveau des contenus des visions, du fait de la singularité temporelle, ces visions ne sont pas uniquement du futur, mais peuvent également être du passé/présent ou d'autres versions de la boucle, information qui n'est pas connue par les joueurs au début de la partie. Les visions sont un moyen pour le MJ de contrôler l'évolution de la partie, typiquement en incluant des indices/informations dans certaines visions. Les joueurs peuvent garder leur carte vision (avec le texte associé), cependant il est interdit de lire une carte vision d'un autre joueur. Il est également fortement conseillé de ne pas lire mot à mot les cartes, mais les reformuler dans ses propres termes.
	
	
	\paragraph{Paradoxes} Un certain nombre de personnages croient qu'un paradoxe peut potentiellement les effacer de la réalité (dans le meilleur des cas) ou détruire l'univers (dans le pire des cas). Les joueurs qui croient aux paradoxes sont:
	\begin{itemize}
		\item \nmPlayerII (trafiquant), ce qui peut poser des situations intéressantes avec \nmPlayerIII
		\item \nmPlayerIII (étudiant)
		\item \nmPlayerV (visiteur du futur), qui ne se souvient pas de s'être rencontré dans le passé, mais ça vient peut-être d'un effacement local de mémoire
		\item \nmPlayerVII (inspecteur), qui se débrouille pour désamorcer les-dits paradoxes.
		\item \nmPlayerVIII (garde du corps)
		\item \nmPlayerX (highlander), qui se fiche si d'autres personne se font effacer de la réalité, du moment que ça ne le concerne pas
		\item \nmPlayerXI (catalyse), qui est au milieu de tout ça et veut aller pleurer tranquillement dans son coin
	\end{itemize}
	
	\par L'inspecteur a pour mission de \emph{désamorcer} des paradoxes, c'est à dire qu'il a pour mission de faire jouer les scènes des visions, afin que ces dernières soient résolues. Pour cela, il a accès à l'holodeck de la station (gérant les décors), mais doit trouver des acteurs pour jouer leur rôle. Certes le contenu de la vision est exposé au grand jour, mais il n'y a plus de paradoxes, vu que la vision n'a plus besoin de forcément se passer.
	
	
	% ==========================================================
	\subsection{Évènements au cours de la murder}
	
	\par La progression de la murder est principalement dictée par l'avancement dans le contenu des visions. Mis à part ça, il y a d'autres évènements qui peuvent se passer pendant la murder:
	\begin{itemize}
		\item \textbf{Début de la murder:} Une alarme sonne, les portes se font scellées et l'IA fait un discours pour résumer la situation, avant que l'inspecteur et son équipe rentre en jeu. Le discours de l'IA est:\\
		\emph{(voix mécanique)} \texttt{ALERTE! ALERTE!}\\
		\texttt{Flux de tachyon détecté dans la zone 25 du relai commercial Aldébaran-41!}\\
		\texttt{Scellage complet du secteur, conformément au protocole\dots Terminé.}\\
		\texttt{Effacement de la mémoire des dernières minutes, conformément au protocole\dots Terminé.}
		
		\emph{(voix plus douce)}
		\texttt{Cher voyageurs en transit, nous vous informons qu'un flux de particule temporel vient d'être détecté dans votre secteur, ce qui implique une consommation illégale d'épice et donc un risque important de paradoxe. Nous vous rappelons que tout paradoxe non résolu peut signifier votre effacement de cette réalité, voire la destruction de l'univers. Heureusement, une équipe de résolution de paradoxe est en chemin et nous vous prions de bien vouloir coopérer avec ces derniers. Le secteur sera descellé une fois que ce dernier jugera les paradoxes désamorcés, et le ou les coupables sécurisés. Encore une fois, nous vous excusons de la gène occasionnée.}
		
		
		\item \textbf{Gâteau à l'épice ($\approx$ 30/45 mins):} \nmPlayerIII (l'étudiant) était en train de cuisiner un gâteau, utilisant (involontairement) de l'épice à la place de la farine. Ce gâteau venait d'être mis au four au moment où l'alerte vient d'être sonné. La cuisson est gérée par des machines automatisées (et donc par l'IA) et cette dernière aura comme instruction de servir une part par personne.
		
		\par Le problème vient de l'algorithme de comptage des personnes, qui tient compte des doublons et quintuplons. De plus, \nmPlayerX n'est pas identifié comme étant humain (suite à toutes les mutations présents dans son corps). Ainsi, uniquement 6 parts de gâteaux (sans épice = une moitié de gâteau) seront distribuées, alors qu'il y a 12 personnes physiquement à bord.
		
		\par Si interrogée, l'IA insistera qu'il y a uniquement "Six personnes à bord, bande de gourmands! Et que le second service ne viendra qu'une fois que vous aurez fini vos parts". Le comptage est fait en "regroupant tous les signes de vie perçu par tous les capteurs dispersés dans la zone, gardant que ceux qui sont humains (afin d'éliminer tout ce qui est insecte et animal) et compter le nombre de différences". 
		
		\par Le gâteau a originellement 12 parts, 6 parts sont distribuées au début, mais 6 autres peuvent. Les personnes mangeant leur part de gâteau auront une vision quelques minutes après (au grand désespoir de l'inspecteur).
		
		
		\item \textbf{Évènements potentiellement récurrents déclenchés par les joueurs:} \nmPlayerVI (la magicienne) et \nmPlayerIX (la psychique) ont des visions de manière récurrente (dû à leurs capacités). \nmPlayerI a une suite de mission visant à s'emparer du corps de \nmPlayerII ou \nmPlayerIII. \nmPlayerXII a un ensemble de quêtes secondaires visant à développer des nouvelles substances chimiques utiles (les tests pouvant avoir des effets assez aléatoires).
		
		\item \textbf{Fin de la murder:} Cf la sous-section précédente, paragraphe "Donneur d'ordre de la fin de l'isolation".
	\end{itemize}
	
	
	\section{Liste des visions}
	
	\par Les visions sont dans l'ordre dans lesquelles elles devraient être distribuées, de faiblement à fortement incriminantes/informatives. Cela permet d'induire une progression dans la murder. Du fait de la malédiction de \nmPlayerXI, toutes les visions le concernant sont généralement plutôt au début, indépendamment de leur niveau d'incrimination.
	
	\begin{enumerate}
		% BAS
		% 2 et 8 - marchandage d'épice (2 est sensé délivré de l'épice à 8, sauf qu'il a paumé son paquet de farine, vu que 3 l'a volé entre temps)
		\item Tu vois \nmPlayerII dans la cuisine, qui cherche frénétiquement quelque chose, ouvrant et refermant des placards, pots\dots Il a l'air de s'affoler de plus en plus et accélère encore plus ses recherches, manquant de faire tomber plusieurs pots. Tout à coup, \nmPlayerVIII ouvre la porte violemment, faisant sursauter \nmPlayerII. Après avoir vérifiés qu'ils étaient bien seuls, \nmPlayerVIII met en joue \nmPlayerII et lui demande "Où es-ce qu'elle est?"
		
		% 11, intervient alors que 1 était en train de convertir 2 et exécute 1. 7 arrive après.
		\item Tu vois \nmPlayerII à terre et évanoui, avec \nmPlayerI à genoux à côté. Une des jambes de \nmPlayerI semble s'être pris un tir de blaster. Au dessus d'eux se dresse \nmPlayerXI, qui lève le blaster dans sa main et met en joue \nmPlayerI. \nmPlayerI ouvre les yeux de surprise et dit rapidement "Non non no-" avant que \nmPlayerXI l'exécute d'un tir entre les deux yeux. A ce moment là, \nmPlayerVII arrive dans la salle et \nmPlayerXI se fige. Après un long silence gêné, \nmPlayerVII dit "Euuuuh\dots Ce n'est vraiment pas ce que vous croyez."
		
		% 5 vient de révéler ce qu'il est à 7 & le potentiel futur apocalyptique
		\item Tu vois \nmPlayerVII qui regarde \nmPlayerV\dots correction: qui fixe d'un regard abasourdi \nmPlayerV, qui est un peu mal à l'aise sous l'attention. \nmPlayerVII rompt le silence gêné "\dots Vraiment?", à quoi \nmPlayerV acquiesce de la tête. "Non, parce que je vois au moins 3 raisons pour lesquelles ce que tu me dis est impossible, sans compter le fait que tout le plan est juste complètement suicidaire". \nmPlayerV garde le silence pendant un moment, avant que l'inspecteur continue "La situation s'était vraiment dégradée à ce point?". Sur quoi \nmPlayerV lui répond "Pire".
		
		% 4 et 11
		\item Tu vois \nmPlayerXI roulé en boule et sanglotant. \nmPlayerIV est assis à côté, sirotant de manière décontractée sa boisson et lui dit "En tout cas, félicitation pour le feu d'artifice. Tu viens probablement de mettre fin à un empire millénaire de manière assez radicale. Et de quelle manière! Tant d'éléments imbriqués, s'entraînant et se combinant entre eux\dots c'était magnifique. Je doute honnêtement que, si c'était mon but, je n'aurais pas pu faire mieux. Enfin si: j'aurais au moins eu la décence de fournir du pop-corn à mes spectateurs". Cela ne semble vraiment pas remonter le moral de \nmPlayerXI, qui sanglote encore plus.
		
		% Réaction de 9 à 11
		\item Tu vois \nmPlayerIX fixer \nmPlayerXI du regard, d'un air intriguée. "Je n'ai jamais vu autant de potentiel chaotique entourant une personne". \nmPlayerXI pousse un soupir agacé "Je sais\dots". \nmPlayerIX l'ignore et continue "C'est comme si quelqu'un avait pris ton fils du destin, et s'était amusé à le mettre en boule, tout en le parsemant de nœuds papillons". \nmPlayerXI lui répond "Oui, je sais! Mais es-ce que ça veut dire que vous pouvez résoudre le problème?".
		
		% Expérience de 12 sur 3 non concluante
		\item Tu vois \nmPlayerIII qui s'approche de \nmPlayerXII de manière hésitante. "Dites\dots au sujet de votre test que vous m'aviez fait tout à l'heure?" \nmPlayerXII: "Ah oui, au sujet de ta consommation d'épice?". \nmPlayerIII a l'air de s'énerver sur cette remarque, mais retient sa langue, laissant \nmPlayerXII finir: "Alors, ça a donné quoi?". \nmPlayerIII lui répond avec un regard noir, avant de soulever son pull, montrant des taches rouges qui couvre la peau de son ventre. "Oh\dots C'est donc un échec".
		
		
		% INTERMÉDIAIRE
		% 6 protégeant 9 face à 10
		\item Tu vois \nmPlayerX qui scie une porte en utilisant des dagues laser. Après quelques secondes à finir son tracé, la porte cède, révélant \nmPlayerVI et \nmPlayerIX. Cette dernière tremble à terre, derrière les jambes de \nmPlayerVI et semble blessée. \nmPlayerVI semble furieuse et proclame "Shoggy? Attaque!". Des grognements de chien surgissent d'un coin de la pièce, mais la vision s'interrompt brutalement avant que tu puisses décerner quoique ce soit.
		
		% 9 geass 8 pour tuer 10
		\item Tu vois \nmPlayerIX regarder \nmPlayerVIII, ce dernier étant avachi dans un coin de la pièce, le regard vide. \nmPlayerIX sort de la pièce, quelques secondes avant que \nmPlayerVIII se réveille en sursaut. Ce dernier semble affolé et, après s'être assuré d'être seul dans sa salle murmure à lui-même "Il faut que j'élimine \nmPlayerX avant qu'il devienne trop dangereux. La survie de l'empire peut en dépendre", alors qu'il sort de la pièce, il passe devant \nmPlayerIX qui a un sourire satisfait au coin.
		
		% 1 donne une seringue à 3 qui efface ses mémoires
		\item Tu vois \nmPlayerI passer une seringue à \nmPlayerIII, en lui disant "Voilà ce qui devrait régler tes problèmes de faim d'ici les prochains jours". \nmPlayerIII semble un peu hésitant, mais accepte finalement et se l'injecte dans le bras. \nmPlayerI quitte la salle, laissant \nmPlayerIII seul, qui semble s'endormir. Quelques minutes plus tard, \nmPlayerIII se réveille d'un sursaut, regarde ses alentours d'un air apeuré, avant de se dire à soit-même "Qu'es-ce qu- Où suis-je?".
		
		% 4 parle à 11 et révèle l'existence de la boucle temporelle
		\item Tu vois \nmPlayerIV parler à \nmPlayerXI "\dots En magouillant ainsi, l'étudiant a fini par tuer l'highlander en combat singulier et s'est fait promettre général de l'empire, le tout contre sa volonté. D'un autre côté, les acteurs étaient tellement prévisibles que c'était comme s'ils demandaient à ce que quelqu'un les manipulent. Ce n'est pas comme toi: d'habitude le chaos qui t'entoures est extrêmement difficile à contrôler et donc très rafraîchissant à observer. "D'habitude?" releva \nmPlayerXI d'un air intrigué. "\dots Oui, c'est la 147ème fois que tout cela c'est déroulé, avec des variations diverses et parfois provoquées. Et je ne parle même pas des multiples versions d'une même personne qui se promène".
		
		
		% HAUT
		% 12 et 7 mettent 9 aux arrêts car fugitive de l'empire. 9 se débat et affirme qu'elle a été illégalement emprisonnée, et qu'elle n'aurait jamais dû laisser ces expériences se mener (car sans elles, cette maudis épice n'aurait jamais existé)
		\item Tu vois \nmPlayerVII plaquer \nmPlayerIX face contre terre, avec \nmPlayerXII à côté, lui disant "\nmPlayerIX, vous êtes en état d'arrestation. Vous serez raccompagné très prochainement dans les laboratoires de l'empire desquels vous vous êtes enfuie". \nmPlayerIX se débat et crache "Je n'aurais jamais dû vous laisser m'étudier. Sans cela, cette maudite épice n'aurait jamais existée".
		
		% Révélation de 12 à 7 comme quoi les paradoxes sont du bidon. 7 craque mentalement, 12 l'abat
		\item Tu vois \nmPlayerXII dressé au dessus de \nmPlayerVII qui est blessé. \nmPlayerVII prend la parole: "Pourquoi? Tu sais aussi bien que moi que c'est la fin et que ces paradoxes ne peuvent plus se faire éviter. Donc pourquoi prendre la peine de tuer systématiquement tous les gens de la zone et d'en créer encore plus par la même occasion?". \nmPlayerXII sourit "Es-ce que tu connais le secret de l'administration de l'empire?". \nmPlayerVII se fige, confus et silencieux.	Le sourire de \nmPlayerXII s'élargit encore plus: "Le grand secret de l'administration de l'empire, c'est que les paradoxes n'existent pas". Après cette révélation, \nmPlayerVII écarquilla des yeux, choqué, avant de pousser un long cri de rage et de folie, qui ne s'arrêta qu'une fois que \nmPlayerXII l'achève d'un coup de blaster.
		
		% 12 tue 6 puis commence à boire son sang... Puis truc se passe mal, incompréhension... "C'est impossible?" avant qu'il vomisse.
		\item Tu vois \nmPlayerXII décapiter violemment \nmPlayerVI avec ses dagues, puis s'agenouiller pour porter le sang qui s'écoule de la blessure à ses lèvres. Après quelques secondes passées à boire, il se relève brusquement, des veines pulsant sur ses joues et ayant l'air très confus. Il fit quelques pas en titubant avant de s'exclamer "Mais\dots C'est impossible!". Puis, il se tordit de douleur et vomit.
		
		% Révélation que 1/2/3/4 sont la même personne par 4 (ce qui fait 4 sur 11\dots)... ce qui est deux fois plus que cet inspecteur et sa version future qui veut empêcher la destruction de l'empire.
		\item Tu vois \nmPlayerII, \nmPlayerI et \nmPlayerIII se disputer et s'entre-pointer du doigt, avant que \nmPlayerIV arrive dans la pièce et dise tout fort "Vous savez que se parler à soi-même est un des premiers signe de la folie?". Ses trois vict-euh, interlocuteurs s'arrêtent et le regardent, ébahis. \nmPlayerIV continue "\dots Bon d'accord, c'est moi qui dit ça, mais nous sommes les mieux placés pour savoir que nous sommes légèrement hypocrites". Le silence continua à régner dans la pièce. "Et encore", continua \nmPlayerIV pas du tout affecté, "nous ne tenons pas la comparaison avec l'inspecteur et sa version future. Aux dernières nouvelles, ils voulaient s'entre-étrangler. Ça ne doit vraiment pas être sain tout ça".
	\end{enumerate}
	
	Ce qui fait (pour l'instant) 14 visions, 12 venant des parts de gâteau à l'épice, 2 pour \nmPlayerIX, du fait de sa nature et une ou deux pour potentiellement \nmPlayerVI. En plus, \nmPlayerIX peut rajouter des visions à tout ça.
}


% TODO: système d'annonce publique ???
%	=> Laisser du papier et un tableau au cours de la partie pour pouvoir faire des annonces pseudo-anonymes

% TODO: idées de persos supplémentaires?
% TODO: playlist ?




% ***********************************************************************************************************
% *** Titre factice pour la tête des fiches de personnages (et histoire de savoir qui est qui pour le MJ) ***
% ***********************************************************************************************************

\players[1]{\begin{center}\textbf{\nmPlayerI}\end{center}}
\players[2]{\begin{center}\textbf{\nmPlayerII}\end{center}}
\players[3]{\begin{center}\textbf{\nmPlayerIII}\end{center}}
\players[4]{\begin{center}\textbf{\nmPlayerIV}\end{center}}
\players[5]{\begin{center}\textbf{\nmPlayerV}\end{center}}
\players[6]{\begin{center}\textbf{\nmPlayerVI}\end{center}}
\players[7]{\begin{center}\textbf{\nmPlayerVII}\end{center}}
\players[8]{\begin{center}\textbf{\nmPlayerVIII}\end{center}}
\players[9]{\begin{center}\textbf{\nmPlayerIX}\end{center}}
\players[10]{\begin{center}\textbf{\nmPlayerX}\end{center}}
\players[11]{\begin{center}\textbf{\nmPlayerXI}\end{center}}
\players[12]{\begin{center}\textbf{\nmPlayerXII}\end{center}}



% ********************************************************************
% *** Description de l'univers - version partielle (tout le monde) ***
% ********************************************************************

\players[1][2][3][4][5][6][7][8][9][10][11][12]{
	\section{Informations officielles}
	\players[6]{																	% Magicienne
		Tu n'as pas la moindre idée de ce qu'il se passe dans cet univers\dots et encore moins d'idée de ce qu'il se passe tout court.
	}
	\players[1][2][3][4][5][7][8][9][10][11][12]{									% Pas la magicienne
		\paragraph{Bienvenue en l'an 3412} Voilà plus d'un millénaire que l'humanité s'est lancée à la conquête des étoiles, à l'aide de vaisseau pouvant aller à des vitesses supra-luminique. L'empire humain couvre désormais une bonne portion de la galaxie et le commerce est en plein essor entre les différents secteurs de cet empire. Quelques espèces aliens ont été rencontrées, mais après des débuts diplomatiques tumultueux, des accords ont été conclus se résumant en gros à "tout le monde reste dans son territoire, avec un libre passage autorisé". Rencontrer un alien est donc rare, mais peut arriver.
		
		\par La murder se passe dans une station de relai commerciale (nommée \emph{Aldébaran-4}), située dans l'espace profond. Cette station est un point de transit entre plusieurs secteurs de l'empire et une IA automatisée contrôle le bon fonctionnement de la station (notamment la gestion des docks, la maintenance de la station, \dots). Des officiers (majoritairement de douane) sont stationnés dans la station, et inspecte régulièrement les cargaisons commerciales pour vérifier la non-présence de produits illégaux.
		
		% Épice
		\paragraph{Qui contrôle l'épice\dots} Il existe une drogue dans cet univers nommée \emph{l'épice}. Comme son nom l'indique, la consommation de cette drogue provoque des visions du futur. Aussi, si quelqu'un en prend en petite quantité et en continu, elle devient capable de prédire les mouvements d'un adversaire une seconde à l'avance, ce qui augmente sa capacité de combat.
		
		\par Cependant, la consommation de cette drogue peut entraîner un \emph{paradoxe temporel}, qui est un risque majeur. En effet, toute personne impliquée dans un paradoxe non résolu risque de se faire effacer de la réalité, dans le meilleur des cas\dots le pire des cas étant (en théorie) la destruction de l'univers. Ainsi, la production et l'utilisation de l'épice est interdite dans tout l'Empire, et son usage est uniquement réservée à l'armée et aux secteurs critiques du gouvernement, sous environnement contrôlé.
		
		% Flux de tachyon + inspecteur de résolution de paradoxe
		\par La consommation d'épice provoque une émission de flux de particules temporelles nommées \emph{tachyon}. Ainsi des détecteurs ont été installés de manière systématique autour de chaque planète, vaisseau et station pour pouvoir identifier une consommation d'épice et donc intervenir.
		
		\par L'intervention est généralement menée par un inspecteur de résolution des paradoxes qui a plusieurs objectifs: identifier le consommateur d'épice et confisquer la drogue, et désamorcer le paradoxe. Le désamorçage de paradoxe se fait généralement en utilisant un \emph{holodeck}, càd une salle qui peut simuler des objets à coup d'hologramme, afin de pouvoir réacter le paradoxe. La force létale n'est autorisée uniquement qu'en dernier recours et devra être justifiée.
		
		\paragraph{La murder: alerte temporelle à Aldébaran-41} Un flux de tachyon a été détecté dans votre secteur de la station, vous enfermant avec d'autres personnes. Un inspecteur de résolution des paradoxes et son équipe est dépêché pour tenter de déterminer ce qu'il s'est passé, et de désamorcer tout paradoxe pouvant survenir.
	}
	
	
	\section{Informations pratiques sur la murder}
	
	\paragraph{Liste des personnages} La liste des personnages présents dans la zone, ainsi que leur fonction déclarée, sont:
	\begin{center}
	\begin{tabular}{|l||l|}
		\hline
		\emph{\nmPlayerVII} & Inspecteur de résolution des paradoxes\\
		\emph{\nmPlayerXII} & Chimiste, assistant de l'inspecteur\\
		\emph{\nmPlayerVIII} & Garde du corps de l'inspecteur\\
		\hline
		\emph{\nmPlayerX} & Voyageur - Mercenaire\\
		\emph{\nmPlayerII} & Voyageur - Marchand\\
		\emph{\nmPlayerXI} & Voyageur\\
		\emph{\nmPlayerI} & Voyageur\\
		\emph{\nmPlayerIX} & Voyageuse\\
		\emph{\nmPlayerVI} & Voyageuse\\
		\emph{\nmPlayerIII} & Voyageur\\
		\emph{\nmPlayerIV} & Voyageur\\
		\emph{\nmPlayerV} & Voyageur\\
		\hline
	\end{tabular}
	\end{center}
	
	\par (Note HRP: le genre des personnes peuvent changer d'une partie à une autre, dépendant des joueurs. Je vous laisse le soin de substituer les pronoms et accord si changement il y a).
	
	\paragraph{Système de combat} Il y a trois niveaux d'attaque:
	\begin{itemize}
		\item \textbf{Faible:} Blesse légèrement un joueur (ou assomme), l'incapacitant temporairement, mais lui laissant aucune séquelle postérieure.
		\item \textbf{Fort:} Blesse gravement, incapacitant un joueur tout en laissant des (potentielles) séquelles. Peut aller jusqu'à le tuer si aucun soin n'est prodigué dans les quelques minutes qui suivent.
		\item \textbf{Létale:} Tue instantanément un joueur.
	\end{itemize}
	\par Il est possible de réduire la puissance de son attaque. Typiquement, si un joueur a la possibilité de faire, par exemple, une attaque létale, il peut décider de réduire la puissance de son attaque à faible ou fort.
	
	\par Par défaut, une attaque à main nue est de force "Faible". Un entraînement au combat peut monter cette attaque à main nue d'un cran. Une arme (typiquement blanche) est généralement une attaque forte. Un phaseur peut aller jusqu'à une attaque létale. La consommation d'épice peut monter d'un cran la dangerosité d'une attaque.
	
	% Attaque surprise
	\par Quand un joueur est attaqué et le sait, il a la possibilité de faire une réaction (typiquement tenter de se défendre, pousser un grand cri, \dots). L'exception est si l'attaque est une attaque surprise. Càd, l'attaque a été faite hors du champs de vision de la cible ET que l'attaquant n'a pas la main sur une de ses armes. Si la cible a un entraînement martial, elle ne peut pas se faire prendre par surprise (vu que ses poings comptent comme une arme au corps à corps).
	
	% Répercutions judiciaire
	\par Les règles judiciaire classique de l'empire s'appliquent toujours. Du coup, s'il y a meurtre sans bonne raison apparente et que le coupable est identifié, il y aura des répercussions judiciaires par la suite. Tuer en défense est acceptable, si vous arrivez à prouver que vous étiez bien dans ce genre de situation.
	
	% Armement en évidence des joueurs
	\par Il est publiquement connu que les personnes suivantes sont (ou ont été) armées:
	\begin{itemize}
		\item \nmPlayerVIII, en tant que garde du corps, possède un phaseur (puissance létale), et un entraînement au combat (puissance fort à mains nues)
		\item \nmPlayerX, en tant que mercenaire, a une licence pour posséder un phaseur, mais s'en est fait dépossédé très rapidement une fois l'alerte donné, conformément au protocole. En plus de ce phaseur, on lui a également retiré une épée longue (en titanium) et une dague (à vibrations). Il est cependant entraîné au combat (puissance fort à mains nues).
		\item \nmPlayerXI semble avoir eu de l'entraînement au combat à mains nues, mais n'était pas armé.
	\end{itemize}
	
	
	\paragraph{Système des visions du futur} Si une personne a une vision au cours de la partie, il se passe les choses suivantes. Niveau RP, le personnage sent une vision arriver quelques secondes avant cette dernière (typiquement quelques minutes après la consommation d'épice). Puis, il a une vision qui prend une minute, où le personnage est figé et ne peut pas répondre à des stimuli extérieur. Au bout d'une minute, la vision se termine et le personnage revient à lui, avec les informations de sa vision.
	
	\par Au niveau mécanique, le MJ fera un signe discret à un joueur s'il a une vision (tapotage d'épaule, passage de carte vision en douce) et lui donnera une carte vision décrivant la vision. Le joueur a quelques secondes pour s'excuser avant que sa vision frappe et doit compter 1 minutes pile sans qu'il puisse réagir avant de pouvoir reprendre la partie normalement. Il conserve la carte vision avec lui, mais il est interdit de la montrer ou de la lire directement à quelqu'un d'autre. Le joueur est donc obligé de reformuler avec ses propres mots le contenu de la carte, s'il veut en transmettre les informations.
	
	\par Par ailleurs, les visions décrivent une scène (émotionnellement forte) du futur. Elles révèlent donc systématiquement des informations sur des joueurs, et ne peuvent pas se faire données à l'identique à un autre joueur.
}



% ****************************************************************************************************************************************************
% ****************************************************************************************************************************************************
% ****************************************************************************************************************************************************


% ***********************
% *** Fiche de persos ***
% ***********************


%1) Savant Fou
\pageForPlayer{1}{\nmPlayerI}{
	\item[Nom] \nmPlayerI
	\item[Stéréotype] Savant fou
	\item[Objectif] Atteindre l'immortalité en transférant sa conscience dans son corps passé
	\item[Atout majeur] A déjà vécu une fois cette situation (même si c'est flou)
	\item[Inconvénient majeur] Moralité de son objectif discutable - processus de conversion risqué
	\item[Possession] Un kit de conversion (des tests ADN, 2 fioles et une pilule). Machine à voyager dans le temps (inactive).
	\item[Combat] Aucun entraînement
	\item[Compétences] Compétent dans tous les domaines scientifiques. Expert en physique fondamentale et biologie, surtout ce qui touche au temps et à la mémoire.
}{
	\paragraph{Histoire personnelle} Ton nom est \nmPlayerI, et tu es un scientifique de génie. Ton but ultime est de mettre le point final à la recherche scientifique en finissant de répondre à toutes les questions posées par la science, finissant de lever le voile de l'inconnu sur le monde qui nous entoure et menant l'espèce humaine vers un âge d'or éternel. Pour cela, tu as consacré ta vie à la recherche scientifique, quelque soit le domaine, et tu as fait de nombreuses découvertes, te donnant une réputation de successeur de Da Vinci.
	
	\par Cependant, les années passent et malgré tes nombreux succès, tu te rends compte que tu ne vas pas réussir à finir tout tes projets (par manque de temps). Tu as essayé de tacler ce problème en menant des recherches poussées sur la conscience humaine (pour voir si tu pouvais transférer ta conscience dans un corps artificiel), puis sur la manipulation temporelle (pour voir si tu pouvais renverser les effets du passage du temps), mais aucune solution n'était complètement satisfaisante sur le long terme.
	
	\par C'est alors que tu as eu une idée de génie (comme cela est ton habitude). Si tu remontes le temps et que tu transfères ta conscience dans ton corps du passé, tu pourra recommencer ta vie avec toutes les découvertes que tu as déjà faites, et donc aller encore plus loin dans la bataille. Tes recherches ont montré que les paradoxes temporelles n'existent pas (théoriquement, un univers parallèle est créé à chaque remontée dans le temps), ainsi il n'y a strictement aucun risque de s'effacer accidentellement de l'univers. En bonus, si ton plan marche une fois, rien ne t'empêche de le refaire dans quelques années, puis quelques années après, \dots te rendant immortel à l'intérieur de ta boucle temporelle.
	
	\par Éthiquement, tu es conscient que ton plan est discutable. Il s'agit de ton corps, donc personne d'autre subira les conséquences, et vu les avantages que ça apporte pour l'humanité, tu restes convaincu que c'est la meilleure course possible. Néanmoins, tu préfères agir de manière discrète pour éviter de (longues) explications et bâtons dans les roues. De plus, le voyage temporel est très mal vu par l'empire, du fait des abus potentiels.
	
	\par A propos de ta destination, tu te souviens avoir vécu une alerte temporelle quand tu étais jeune, dans la station Aldébaran-41. Du coup, tu as visé ce moment là, ce qui permettra de camoufler le flux de tachyon venant de la déstabilisation temporelle provoqué par ton voyage.
	
	\par C'est à ce moment là que tu t'es rendu compte que tu avais des problèmes de mémoire. Ton hypothèse principale est que quelque chose s'est passé pendant l'alerte, ce qui a fait que les autorités de l'empire ont dû effacer ta mémoire des événements, pour éviter un de leur soi-disant "paradoxe".
	
	% Note: fusion des backgrounds de 2 et 3, du fait de la confusion entre les 2
	\par De ce que tu te souviens très vaguement, tu étais étudiant et cherchait désespérément des sources de financement pour tes travaux. Pour cela, tu t'es mis au trafic d'épice à mi-temps et notamment devait contacter l'un de tes clients dans la station, au moment de l'alerte. Tu sais également que tu avais une fausse identité, mais tu ne sais pas laquelle. Tu te souviens également de quelqu'un qui s'appelait \nmPlayerXI, mais tu n'as aucune idée de pourquoi. Tu sais qu'il y avait un déluge de prédictions, sans savoir ce quelles étaient, ni comment autant ont pu se produire.
	
	\par Finalement, tu as une machine à voyager dans le temps de poche, capable de te faire avancer/reculer de plusieurs années. Cette dernière a juste été configurée pour l'aller, mais tu n'as pas prévu le retour (vu qu'a priori tu en auras pas besoin). Cependant, au cas où tu en aurais besoin plus tard, tu as gardé ta machine (actuellement vidée de son énergie).
	
	\paragraph{Quête personnelle: transfert de conscience} Le processus de transfert de conscience se fait en plusieurs étapes:
	\begin{itemize}
		\item \textbf{Identification du corps:} ton premier objectif est de t'identifier, via test d'ADN. Mécaniquement, il faut que tu récupères un échantillon (typiquement un cheveux, un verre bu par la cible, etc etc) et le file au MJ, qui te donne le résultat de l'analyse au bout de environ 10 minutes.
		
		\item \textbf{Normalisation des signaux neuronaux:} Fiole de liquide qui restructure les connections neuronales, afin de faire correspondre la structure générale avec les tiennes. En pratique, il faut que tu fasses boire un flacon à ta cible, ce dernier aura un mal de tête épouvantable pendant 5 minutes, mais pensera beaucoup plus clairement après coup.
		
		\item \textbf{Effacement complet de mémoire:} Fiole de liquide qui efface la mémoire. Met quelques minutes pour agir. Mécaniquement, le joueur qui se fait effacer sa mémoire perd sa fiche de personnage et ne se souvient plus de rien.
		
		\item \textbf{Transfert de conscience:} Pilule à faire avaler. Met quelques minutes à agir. Mécaniquement, si toutes les conditions sont vérifiées, le joueur ciblé récupère une copie de ta fiche de personnage et joue avec toi, en tant que ton alter-égo. Il ne reste plus qu'à s'assurer que ce nouveau corps survive et ne perde pas ses mémoires.
	\end{itemize}
}


%2) Trafiquant
\pageForPlayer{2}{\nmPlayerII}{
	\item[Nom] \nmPlayerII
	\item[Identité secrète] \nmPlayerIII
	\item[Stéréotype] Trafiquant
	
	\item[Objectif] Réussir à ne pas se faire griller par l'empire - optionnellement réussir sa vente - investiguer (à distance) ce qui lui est arrivé la première fois
	\item[Atout majeur] A déjà vécu une fois cette situation
	\item[Inconvénient majeur] N'a rien compris à ce qui lui est arrivé, pour se retrouver par accident dans le passé.
	
	\item[Possessions] Nada
	\item[Combat] Aucun entraînement
}{
	\paragraph{Histoire personnelle} Tu t'appelles \nmPlayerII et c'est la seconde fois que tu vit cette alerte temporelle dans la station Aldébaran. La première fois, tu t'appelais \nmPlayerIII et, suite à un incident au cours de l'alerte, tu t'es fait projeté 5 ans dans le passé.
	
	% Première itération
	\par Tu n'as pas compris grand chose de ta première itération. A ce moment, tu étais un simple étudiant assez fauché, qui attendait juste que l'alerte se passe et qui cherchait à régler ses problèmes d'argent. Vu que ça fait 5 ans de ton point de vue, tes souvenirs sont un peu floutés, mais voilà ce qui s'est passé:
	\begin{itemize}
		\item Au milieu de l'alerte et pendant l'enquête, une vague de visions est arrivée, faisant désespérer l'inspecteur, qui a dû mettre les bouchées doubles pour désamorcer les potentiels paradoxes.
		
		\item Tu te souviens que \nmPlayerXI avait un rôle de coordinateur assez central dans la résolution de plusieurs affaires. Tu ne t'es que vaguement intéressé à ces dernières et donc ne connaît pas les détails (de mémoire, tu avais passé le temps en cuisinant). Tu sais donc vaguement qu'il y avait des histoires de trafic d'épice sur lequel l'inspecteur était en train d'enquêter, d'évadé d'un centre d'expérimentation de l'empire et de fin du monde apocalyptique dont le paradoxe était compliqué à désamorcer.
		
		\item Tu n'avais eu presque aucun contact avec ta version du futur (maintenant que tu sais qui il est), probablement pour éviter les paradoxes.
		
		\item \nmPlayerI t'avait parlé plusieurs fois et t'avait finalement proposé une série d'expérience pour améliorer tes capacités mentales, afin de t'aider dans tes études. Après coup, tu te rends compte que son intérêt était très louche et tu as décidé d'investiguer ses affaires. Tu es tombé sur une étrange machine dans une de ses poches qui, une fois activée et rechargée partiellement, t'a propulsé d'un coup 5 ans dans le passé, dans une planète agricole lointaine.
	\end{itemize}
	
	\par Après un bref (long) instant de panique, et ne voulant pas te faire emprisonner par l'empire, tu as réussi à prendre contact avec les contrebandiers locaux pour te forger une nouvelle identité. Tu t'appelles donc maintenant \nmPlayerII (un simple anagramme de ton nom original). Cependant, ton changement d'identité était loin d'être gratuit et, pour payer ta dette, tu as dû devenir trafiquant d'épice. Ainsi, tu voyages sous la couverture d'un marchant de nourriture organique, amenant tes produits des mondes fermiers vers des stations commerciales d'autres secteurs, et en glissant des cargaisons d'épice discrètement.
	
	
	\paragraph{Aldébaran-41, encore une fois} C'est la seconde fois que tu vit cette alerte temporelle, et tu es déterminé à découvrir ce qui s'est vraiment passé la première fois. Le problème est que tu ne peux pas contacter directement ta version passé, sans risquer de paradoxe, du coup il va falloir être subtil. De plus, il va falloir t'assurer que le toi de ton passé revive a peu près la même chose dont tu te souviens.
	
	\par En parallèle, tu as toujours tes problèmes d'argent à résoudre, qui seront a priori réglé dès que la cargaison d'épice sera délivré. Ton contact est sensé être une personne infiltrée dans l'administration de l'empire, connaît ton identité et est sensé te contacter. Tu es sensé lui donner les directions pour arriver à l'emplacement où tu as planqué ta livraison d'épice, c'est à dire, un paquet de farine dans un des entrepôts de la station.
	
	\par Une fois la livraison faite et ton alter-égo du passé parti, tu comptes reprendre ton identité originelle in fine, et te servir de ta nouvelle source d'argent pour financer tes études et payer tes recherches.
}



%3) Etudiant
\pageForPlayer{3}{\nmPlayerIII}{
	\item[Nom] \nmPlayerIII
	\item[Stéréotype] Étudiant
	
	\item[Objectif] Régler tes problèmes financiers, par exemple en récupérant un travail
	\item[Atout majeur] Jeune et innocent (ça compte comme atout?)
	\item[Inconvénient majeur] Fauché
	
	\item[Possessions] Des miettes de pain au fond de tes po... ah non désolé, c'était ton repas de hier soir
	\item[Combat] Aucun entraînement
}{
	\paragraph{Histoire personnelle} Ton nom est \nmPlayerIII et tu es actuellement un étudiant, et ton but est d'en apprendre le plus possible sur toutes les sciences. Tu es donc actuellement en train de finir tes études en physique fondamentale (après avoir fini tes études en biologie, et en mathématiques).
	
	\par Le problème est qu'à force de changer de filière, ta bourse d'étude a expiré et tu as du mal à trouver de l'argent pour survivre et financer tes expériences. Heureusement, tu as réussi à trouver un travail d'été pour tenir au moins quelques mois, et tu es donc actuellement chargé de l'entretien de la station commerciale.
	
	\par Tout se passait bien aujourd'hui. Tu as réussi à finir tes tâches d'entretien (du circuit d'aération, du circuit électrique, \dots) relativement rapidement, tu as pu observer une collision de comètes à quelques années lumières d'ici. Tu as eu même le temps de préparer un gâteau, en utilisant des ingrédients "libérés" d'un entrepôt, qui devrait pouvoir durer pendant un jour. Cependant, l'alerte a sonné pile quand tu as mis le gâteau au four. La cuisson est automatisé par la station, du coup pas de risque de sur-cuisson, mais ça risque de vouloir dire qu'il va falloir que tu partages ton gâteau. Par contre, tu n'as pas eu le temps de ranger tout tes ingrédients et le gâteau devrait être près d'ici une demi-heure.
	
	\par A propos de l'enquête et de la consommation d'épice, tu n'as strictement rien à te reprocher. Du coup, a priori, tu n'as juste à attendre que ça passe, en évitant de t'attirer des ennuis. Au meilleur des cas, et avec un peu de motivation, ce serait bien que tu récupères un peu d'argent (si tu en as l'occasion). Par exemple, en décrochant un travail qui paye mieux et plus intéressant scientifiquement. Typiquement, te faire embaucher par l'agence de détection des paradoxes te plairait bien, mais il va falloir te mettre dans leur bonne grâce.
	
	% Note: pas de référence à 11 ici...
}



%4) Troll immortel
\pageForPlayer{4}{\nmPlayerIV}{
	\item[Nom] \nmPlayerIV
	\item[Identité secrète] \nmPlayerI, mais ce n'est probablement plus pertinent
	\item[Stéréotype] Troll immortel
	
	\item[Objectif] S'amuser, en semant le chaooooos
	\item[Atout majeur] A vécu différentes versions de cette situation des milliers de fois
	\item[Inconvénient majeur] N'est pas complètement sûr de quelle version des personnages il s'agit - Fou
	\item[Pouvoir] Ne peut pas mourir (se régénère dans une dimension parallèle)
	\item[Possessions] Accès à une poche dimensionnelle lui permettant de synthétiser a peu près n'importe quel objet, moyennant un délai
	\item[Entraînement au combat] Aucun
}{
	\paragraph{Histoire personnelle} Ton nom est\dots en fait tu n'es pas certain de quel est ton nom depuis le temps. Tu aimes bien te faire appeler "Chaos" (surtout quand tu es d'humeur dramatique), mais les gens t’appellent plus souvent "That son of a b-". Mis à part ça, \nmPlayerIV est l'identité que tu utilises le plus fréquemment ces derniers siècles.
	
	\par La chose la plus importante à savoir à ton sujet est que tu es immortel. Cela vient d'un plan compliqué que tu avais mis en place quand tu t'appelais \nmPlayerI, ou du moins une de ses versions possibles. Plus précisément, tu as réussi à revenir dans le temps et transférer ta conscience dans ton corps passé, te permettant d'accumuler tes expériences, et de boucler temporellement sans jamais mourir. Depuis, dû aux innombrables avancées scientifiques que tu as mené, tu as trouvé des moyens beaucoup moins contraignant de maintenir ton immortalité, et tu en es au stade où ta mort n'est qu'un léger inconvénient.
	
	\par La seconde chose à savoir à ton sujet est que tu es enfermé dans une boucle temporelle, sans jamais être arrivé à en sortir (tu en es maintenant à la 5423 fois). Tu penses que c'est dû au fait que tu as déjà bouclé des milliers de fois avant que tu trouves un moyen alternatif de vivre éternellement. A chaque début de cycle, tu te retrouves dans cette station, au tout début de l'alerte temporelle.
	
	\par Oh, et à propos des paradoxes, ça fait des dizaines de millénaires que tu sais que c'est du bidon, probablement une propagande de l'empire qui voulait limiter l'utilisation de l'épice. Du coup, quoique tu fasses, tu n'as aucun risque d'effacer ton existence, voire (à ton grand regret) faire exploser l'univers. Aussi, tu as remarqué un phénomène intéressant dû à l'anomalie temporelle locale: les visions (supposées être du futur) peuvent montrer à la place un univers alternatif, ou le passé.
	
	\par Ton objectif principal est de t'amuser. En effet, après avoir vécu autant de fois cette situation, il faut bien faire quelque chose, sans quoi tu risques de sombrer (encore plus) dans la folie. Le mieux serait que tu réussisses à sortir de la boucle temporelle dans laquelle tu es enfermé, mais tu l'as analysé des milliers de fois et as tenté absolument TOUT (y compris les solutions les plus débiles comme plonger dans un trou noir), et ça n'a rien changé. Du coup, tu n'as quasiment aucun espoir que ça arrive.
	
	% Note: sortir de la boucle est peut-être possible en utilisant le portail dimensionnel de 6, vu qu'il change d'univers
	
	
	\paragraph{Archétypes possibles pour chaque personnages}
	\begin{center}
	\begin{tabular}{|l||l|}
		\hline
		\emph{\nmPlayerVII} & Inspecteur honnête - Trafiquant d'épice\\
		\emph{\nmPlayerXII} & Fabriquant d'épice (cherche à l'améliorer) - Comploteur de l'empire\\
		\emph{\nmPlayerVIII} & Robot terminator - Dealer d'épice\\
		\emph{\nmPlayerX} & Agent secret de l'empire - Highlander de l'espace - Empereur en vacances\\
		\emph{\nmPlayerII} & Trafiquant - Fils caché de l'empereur\\
		\emph{\nmPlayerXI} & Catalyse - Demi-dieu du destin (investigue ce bazar) - autre version de toi\\
		\emph{\nmPlayerI} & Alien infiltré - Grand prêtre du destin - Savant fou\\
		\emph{\nmPlayerIX} & Élue de la prophétie (quel qu'elle soit) - Prophète psychique\\
		\emph{\nmPlayerVI} & Euh... aucune idée? (comment c'est possible?)\\
		\emph{\nmPlayerIII} & Tueur à gage (visant \nmPlayerX) - Étudiant inoffensif\\
		\emph{\nmPlayerIV} & C'est toi!\\
		\emph{\nmPlayerV} & Visiteur du futur - Robot terminator\\
		\hline
	\end{tabular}
	\end{center}
	\par Concernant \nmPlayerVI, c'est la première boucle que tu la vois. Ça aurait dû être impossible, vu le nombre considérable d'itération que tu as fait, et que tu as dû déjà avoir exploré toutes les possibilités. Un peu de nouveauté sera bon pour ton amusement et tu as hâte d'analyser cette nouvelle venue.
	
	\par Dépendant de l'itération, certaines personnes sont en fait les mêmes, à des points temporels différents. Par exemple, \nmPlayerV est fréquemment une version future de \nmPlayerVII, ou, au minimum, une version venant du futur de \nmPlayerVII et qui cible ce dernier. \nmPlayerII et \nmPlayerIII sont également des fois la même personne, tout comme \nmPlayerX et \nmPlayerVIII ou \nmPlayerXII et \nmPlayerIII.
	
	\par \nmPlayerXI a été fréquemment la cible de pas mal d'événements. Des fois, c'est parce qu'il était maudit, des fois c'est parce qu'il provoque les-dits événements. Dans tous les cas, c'est une personne que le chaos suit de très près, et donc intéressante à garder à l'œil, voir à s'amuser avec.
	
	\paragraph{Mécanisme de synthèse d'objet} Possibilité de l'utiliser 2 fois durant la partie. Le MJ te donne un bout de carton, un marqueur et une paire de ciseau, et le temps que ça te prend à synthétiser l'objet correspond au temps que ça te prend à dessiner, puis découper le carton pour fabriquer l'objet.
	
	\par Typiquement, un des objets que tu fabriques fréquemment est une main taser (mécaniquement, serrer ta main peut assommer ta cible par surprise).
}



%5) Visiteur du futur
\pageForPlayer{5}{\nmPlayerV}{
	\item[Nom] \nmPlayerV
	\item[Identité secrète] \nmPlayerVII
	\item[Stéréotype] Visiteur du futur
	
	\item[Objectif] 
	\item[Atout majeur] 
	\item[Inconvénient majeur] 
	\item[Possessions] 
	\item[Entraînement au combat] 
}{
	\paragraph{Histoire personnelle} 
	
	
	% TODO
	
	% TODO: a littéralement improvisé sur nom nom (L- euh Luc .... Estravos? (vu qu'il est en "extra") )
	% TODO: objectif: éviter la création d'un "super-méchant" (avec des pouvoirs divins) qui apparemment a été créé durant le lockdown et qui a pris le contrôle de l'empire et détruit le reste
	
	% TODO: sait qu'il n'a pas conservé les souvenirs en tant que 7 => effacement de la mémoire (2 jours après). But: se sacrifier temporellement pour éliminer la menace
	
}



%6) Magicienne
\pageForPlayer{6}{\nmPlayerVI}{
	\item[Nom] \nmPlayerVI
	\item[Stéréotype] Magicienne
	
	\item[Objectif] 
	\item[Atout majeur] 
	\item[Inconvénient majeur] 
	\item[Possessions] 
	\item[Entraînement au combat] 
}{
	\paragraph{Histoire personnelle}
	
	
	
	% TODO
	
	% TODO: y a des voyants qui clignotent, et tu ne penses pas que c'est au mana...
	
	% TODO: technologiste sont des gens dangereux, mieux vaut pas te déclarer comme magicienne
	
	% TODO: peut avoir des prophéties subites (rare)
	% TODO: capable de détecter (et résister) l'utilisation de pouvoir psy de \nmPlayerIX
	
	% TODO: ne croit pas aux paradoxes (en tout cas, dans son univers. Aucune idée dans celui-ci)
	
	% Sorts:
	
	% TODO: capacité magique: sent la magie/truc surnaturel qui se passe
	\paragraph{Sorts:} Tu as une limite de 3 lancements de sorts durant la partie. Pour lancer un sort, tu dois prononcer à voix clair (càd, sans chuchoter, mais pas forcément tout fort) une suite de syllabes associé au sort, tout en pointant du doigt ta cible. Les sorts qui te sont disponibles sont:
	\begin{itemize}
		\item Sommeil () : Assomme un personnage pendant 3 minutes.
		\item % TODO
		
		% TODO: sorts pas de combat?
		
		\item Shoggy (\emph{Fn'agth Radngan Traelfcre Scrortrac}): Invoque un bébé shoggoth plein de tentacules et trôôôôôp mignon. Tu l'as rencontré suite à une expérience (raté) de portail dimensionnel, et depuis il ne veut plus te lâcher. Shoggy est très affectueux et protège quiconque peut potentiellement menacer sa maman (avec forcément une force létale).
	\end{itemize}
	
	
	
	% TODO: condition de victoire: sort de retour
	
}



%7) Inspecteur
\pageForPlayer{7}{\nmPlayerVII}{
	\item[Nom] \nmPlayerVII
	\item[Stéréotype] Inspecteur
	
	\item[Objectif] 
	\item[Atout majeur] 
	\item[Inconvénient majeur] 
	\item[Possessions] 
	\item[Entraînement au combat] 
}{
	\paragraph{Histoire personnelle}
	
	% TODO
	
	% TODO: il est avertit à chaque fois qu'il y a une anomalie temporelle (marche via SMS?)
	
	% TODO: a une salle holographique à sa disposition (voir ce qu'on peut/ne peut pas simuler)
	% Possède un taser (assome, décharge faible)
}




%8) Agent de sécurité
\pageForPlayer{8}{\nmPlayerVIII}{
	\item[Nom] \nmPlayerVIII
	\item[Stéréotype] Agent de sécurité - dealer d'épice
	
	\item[Objectif] 
	\item[Atout majeur] 
	\item[Inconvénient majeur] 
	\item[Possessions] 
	\item[Entraînement au combat] 
}{
	\paragraph{Histoire personnelle}
	
	% TODO
	% Possède un phaseur
	
	% TODO: but => récupérer la cargaison d'épice (si possible sans régler) / ne pas se faire chopper / de manière générale, s'enrichir
	
}


%9) Psychique
\pageForPlayer{9}{\nmPlayerIX}{
	\item[Nom] \nmPlayerIX
	\item[Stéréotype] Prophète psychique
	\item[Objectif] 
	\item[Atout majeur] 
	\item[Inconvénient majeur] 
	\item[Possessions] 
	\item[Entraînement au combat] 
}{
	\paragraph{Histoire personnelle}
	
	% TODO
	
	% TODO: couverture: voyageuse, veut rentrer voir sa famille à [station agricole paumée]
	
	% TODO: origine de l'épice (du fait de l'analyse de sa biologie)/ actuellement pourchassée par l'empire
	% TODO: a des visions assez régulièrement, est toujours HS pendant les visions (du fait de la transmission d'info), mais n'a pas besoin de flux de tachyon et peut contrôler le moment du déclanchement
	% TODO: possibilité de stocker jusqu'à une dose d'épice dans son corps pour soit déclancher une vision, soit transmettre une vision (qu'elle créé) à un joueur (qui aura les symptomes normales de l'épice, mais sans flux de tachyon)
	
	% ( TODO: possibilité de démaudir quelqu'un en claquant une charge)
	
	% TODO: ne croit pas aux paradoxes (les a déjà violé maintes fois)
	
}



%10) Highlander de l'espace
\pageForPlayer{10}{\nmPlayerX}{
	\item[Nom] \nmPlayerX
	\item[Stéréotype] Highlander de l'espace
	
	\item[Objectif] 
	\item[Atout majeur] 
	\item[Inconvénient majeur] 
	\item[Possessions] 
	\item[Entraînement au combat] 
}{
	\paragraph{Histoire personnelle}
	
	% TODO
	
	% TODO: mercenaire voyageant
	
	% TODO: But: ascender à la divinité
	% Possède des dagues planquées sur lui (reste des armes qu'on ne lui a pas pris)
	% Est un peu furax qu'on lui a piqué son phaseur
}


%11) Catalyste
\pageForPlayer{11}{\nmPlayerXI}{
	\item[Nom] \nmPlayerXI
	\item[Stéréotype] Catalyse
	
	\item[Objectif] Éviter les problèmes (mais ce ne serait pas réaliste) - Survivre (ce qui est déjà plus réaliste) - Arriver à réduire/enlever ta malédiction
	\item[Atout majeur] Des choses vont t'arriver\dots
	\item[Inconvénient majeur] Des choses vont t'arriver\dots
	\item[Possessions] Un besoin insatisfait d'avoir des vacances tranquilles
	\item[Entraînement au combat] Moyen (à force de t'échapper de situation potentiellement apocalyptique, tu as réussi à apprendre deux ou trois trucs)
}{
	\paragraph{Histoire personnelle} Ton nom est \nmPlayerXI et tu es maudit. Tu supposes qu'un de tes ancêtres a énervé la mauvaise personne et toute ta lignée s'est retrouvée sous le joug de la (célèbre) malédiction "Que votre vie soit intéressante".
	
	\par Ainsi, depuis ton adolescence, tu t'es retrouvé dans des situations extrêmement improbable, aux enjeux extrêmes. Par exemple, si l'administrateur d'une planète perd ses codes de ses missiles anti-matières, devinez qui les trouve par accident, avec 2 groupes terroristes et des agents de sécurités qui cherchent à les récupérer à tout prix? Ou alors, une anomalie spatiale survient dans un secteur, et deviner qui se trouve dans le vaisseau qui se la prend en première ligne?
	
	\par Heureusement, à force d'éviter les situations apocalyptique, tu as commencé à avoir suffisamment d'entraînement physique pour être jugé apte au combat (c'était ça ou mourir dévoré par des tulipes géantes\dots et non, ce sont des mémoires répressées dont tu n'as pas envie d'en parler).
	
	\par Bien sûr, aujourd'hui n'échappe pas à la règle: quelques minutes à peine après avoir débarqué dans la station, une alerte temporelle s'est déclenchée et tu te retrouves en quarantaine pendant que l'enquête se poursuit. La situation te semble trop fade comparé à ce dont tu as l'habitude, du coup tu es quasi que quoiqu'il se passe, la situation va s'empirer.
	
	\par Ton but principal est donc de survivre (physiquement et mentalement) aux situations de folies qui vont très certainement te tomber dessus. Après, si, par une chance incroyable, tu arrives à te débarrasser de ta malédiction, tu ne diras pas non\dots mais il faudrait une sacrée chance (et un individu compétent) pour que ça arrive.
}
% Note: malédiction enlevable par 6 (sort) ou 9 (charge)



%12) Chimiste - loyal à l'empire
\pageForPlayer{12}{\nmPlayerXII}{
	\item[Nom] \nmPlayerXII
	\item[Stéréotype] Chimiste expérimental - Loyaliste de l'empire
	
	\item[Objectif] 
	\item[Atout majeur] 
	\item[Inconvénient majeur] 
	\item[Possessions] 
	\item[Entraînement au combat] 
}{
	\paragraph{Histoire personnelle}
	
	% TODO
	
	% TODO: sait que paradoxes est intox de l'empire
	
	% TODO: loyal à l'empire, dans l'équipe de l'inspecteur
	% TODO: priorité (dans l'ordre): faire tenir l'intox sur les paradoxes / aider l'inspecteur / conclure ses recherches
	% TODO: priorité 1 => peut justifier un meurtre du point de vue de l'administration de l'empire (peut magouiller), même si non préférable. Sensé prendre le relai sur inspecteur si problème de ce genre et se rendent compte que paradoxe est intox
	
	\paragraph{Recherches} 
	\begin{itemize}
		\item Immunisation contre l'épice
		
		\item Etude génomique d'anomalie
		
		\item Omniscience localisée
	\end{itemize}
	% TODO: besoin de sujets de test pour pouvoir débloquer ces recherches
	
	
	% TODO: possède également une drogue d'effacement locale de mémoire (sur 2 jours, prend 1h à se mettre en place, efface les 2 derniers jours)
	
	% TODO: matos => possibilité d'utiliser l'épice pour booster le combat
}


\end{document}

